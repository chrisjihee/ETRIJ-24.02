\documentclass[AMS,STIX2COL]{WileyNJD-v2}

\articletype{Article Type}%

\received{26 April 2016}
\revised{6 June 2016}
\accepted{6 June 2016}

\raggedbottom


\begin{document}

    \title{Improvement of Korean Morphological Analysis System Through Transformer-based Re-ranking}

    \author[1]{Author One*}

    \author[2,3]{Author Two}

    \author[3]{Author Three}

    \authormark{AUTHOR ONE \textsc{et al}}


    \address[1]{\orgdiv{Org Division}, \orgname{Org Name}, \orgaddress{\state{State name}, \country{Country name}}}

    \address[2]{\orgdiv{Org Division}, \orgname{Org Name}, \orgaddress{\state{State name}, \country{Country name}}}

    \address[3]{\orgdiv{Org Division}, \orgname{Org Name}, \orgaddress{\state{State name}, \country{Country name}}}

    \corres{*Corresponding author name, This is sample corresponding address. \email{authorone@gmail.com}}

    \presentaddress{This is sample for present address text this is sample for present address text}

    \abstract[Abstract]{
        Korean morphological analysis plays a basic and important role enough to be called the first step in Korean language analysis.
        Due to the nature of Korean agglutinative words, it was difficult to build an automatic analysis system because the analysis was not completed with part-of-speech tagging alone, as in English.
        In addition, various methods for morphological analysis have been proposed, but efficient methods such as BPE are mainly used in applications intended to be used as tokenizers for deep learning.
        In this paper, we propose a method to maximize the performance of an efficient morphological analysis system that can be used for tokenization through multi-stage re-ranking based on deep learning.
        For a number of various cases whose rankings have been changed through re-ranking, it is possible to improve performance while maintaining speed by updating the cost matrix of the lattice-based morphological analysis system in the future.
        Through experiments, we showed that the proposed method effectively improved performance by reducing errors by more than 40\% in both the spoken language model and the written language model.
    %한국어 형태소 분석은 한국어 분석의 첫걸음이라고 할 정도로 기본적이고 중요한 역할을 담당하고 있다.
    %한국어는 교착어의 특성으로 인해 영어처럼 품사 태깅만으로는 분석이 끝나지 않아 자동 분석 시스템 구축에 어려움이 있어왔다.
    %또한 형태소분석을 위한 다양한 방법들이 제안되었지만, 딥러닝을 위한 토큰화기로 활용하려는 응용에서는 BPE와 같은 효율적인 방법들이 주로 사용되고 있다.
    %본 논문에서는 토큰화에서도 사용이 가능한 효율적인 형태소분석 시스템의 성능을 딥러닝 기반의 Multi-Stage 재순위화를 통해 극대화하는 방법에 대해 제안하고자 한다.
    %재순위화를 통해 순위가 바뀌어진 다수의 다양한 사례들은 추후 Lattice 기반 형태소분석 시스템의 cost matrix를 업데이트함으로 속도를 유지하면서 성능을 개선하는 것이 가능해진다.
    %실험을 통해 제안하는 방법을 통해 구어체 모델과 문어체 모델에서 모두 기존보다 오류가 40% 이상 감소됨 보임으로 제안한 방법이 효과적으로 성능을 개선하였음을 보였다.
    }

    \keywords{keyword1, keyword2, keyword3, keyword4}

    \JELinfo{classification}

    \MSC{Code numbers}

    \jnlcitation{\cname{%
        \author{Williams K.},
        \author{B. Hoskins},
        \author{R. Lee},
        \author{G. Masato}, and \author{T. Woollings}} (\cyear{2016}),
        \ctitle{A regime analysis of Atlantic winter jet variability applied to evaluate HadGEM3-GC2}, \cjournal{Q.J.R. Meteorol. Soc.}, \cvol{2017;00:1--6}.}


    \maketitle

    \footnotetext{\textbf{Abbreviations:} ANA, anti-nuclear antibodies; APC, antigen-presenting cells; IRF, interferon regulatory factor}


    \section{Introduction}\label{sec1}

    Korean morphological analysis is the process of determining parts of speech by finding morphemes, which are the smallest units of language expression that have an independent meaning in a sentence.
    In an isolating language like English, this can be done relatively simply by tagging parts of speech sequentially, but in Korean, the nature of the agglutinative language requires separating endings or investigatives and restoring inflections to their original form.
    In addition, since the basic input of other Korean analysis tasks is often a separated morpheme, the accuracy of morphological analysis greatly affects the performance of Korean analysis.
    Modern, high-performance deep learning methods in natural language processing use a tokenisation process that breaks text into smaller units, and then converts each token into a vector as input to the computational model [Milkolov2013].
    In this case, the token unit is mainly subword units, and to reflect the characteristics of Korean, subword tokenisation is attempted with separated morphemes in advance. [Song2021]
    Using the results of morphological analysis for this tokenisation process improves the overall performance of the analysis by reflecting the semantic units of Korean, but requires a highly accurate and fast morphological analyzer.

    %한국어 형태소 분석은 문장 내에서 독립적 의미를 가진 가장 작은 언어표현 단위인 형태소를 찾아 품사를 결정하는 과정이다.
    %영어와 같은 고립어에서는 기호 및 띄어쓰기로 나뉘어진 단위에 품사 태깅을 하는 것으로 그 과정이 비교적 단순하나, 한국어의 경우 교착어의 특성으로 인해 어미나 조사를 분리해주고 용언 등의 변화형을 원형으로 복원해주는 과정이 필요하다.
    %또한 다른 한국어 분석 태스크들의 기본 입력이 분리된 형태소인 경우가 많아, 형태소 분석의 정확도는 한국어 분석의 성능에 큰 영향을 미치게 된다.
    %자연어처리 분야에 높은 성능을 자랑하는 최신의 딥러닝 방법들에서는 텍스트를 작은 단위로 분리하는 토큰화 과정을 거친 뒤, 각 토큰을 벡터로 변환하여 입력으로 사용한다. [Milkolov2013]
    %이 때의 토큰 단위는 주로 서브워드 단위를 많이 사용하는 추세이고, 한국어의 특성을 반영하여 형태소를 미리 분리한 상태에서 서브워드 토큰화를 시도하기도 한다. [Song2021]
    %이러한 토큰화 과정에 형태소 분석 결과를 사용하게 되면 한국어의 의미 단위를 반영할 수 있어 전반적인 분석 성능이 좋아지지만, 정확도가 높으면서 빠른 형태소 분석기를 필요로 하게 된다.

    Various approaches have been proposed for morphological analysis, which plays a fundamental and important role in Korean language analysis. [...]
    In general, when people understand speech or writing, they try to make sense of it using the vocabulary and concepts they know.
    While there are ways to use rules or dictionaries to reflect this way of understanding [...], the problem is that it becomes difficult to build and maintain a dictionary for vocabulary that appears in every text.
    Hence, methods for tagging in syllable units without a dictionary have been proposed [...], and studies to improve them have been continuously conducted. [...]
    From a mechanical point of view, syllable-by-syllable morphological analysis can be done either by tagging syllable-by-syllable and then applying a base form restoration dictionary [...], or by tagging syllable-by-syllable with the base form already restored. [...]
    However, syllable-by-syllable morphological analysis has limitations in that it is difficult to accurately identify the boundaries of morphemes, and it is difficult to learn long-term contextual information as the length of the sequence increases.
    In this paper, the former is referred to as dictionary-based morphological analysis, and the latter as syllable unit morphological analysis.
    Both methods also share the limitation that they are trained on a manually labelled corpus and cannot perform accurate analysis of new syllable combinations or morphemes that do not appear in the training corpus.
    In recent years, with the development of the Internet and the spread of open source and open data, web texts, corpora, language resources, and knowledge shared by various people have accumulated considerably.
    The reduced cost of building and maintaining a dictionary can be a great opportunity to overcome the limitations of dictionary-based methods.

    %이렇게 한국어 분석에서 기본적이고 중요한 역할을 담당해오는 형태소 분석에 대해서 다양한 접근법들이 제안되어 왔다.
    %일반적으로 사람들은 말이나 글을 이해할 때 자신이 알고 있는 어휘와 개념들을 이용해서 그것을 이해하려고 한다.
    %이와 같은 사람들의 이해 방식을 따라 규칙이나 사전을 사용한 방법들이 있었지만, 모든 텍스트에서 등장하는 어휘에 대한 사전을 구축하고 유지하는 것이 어렵게 되는 문제가 있었다.
    %이에 사전 없이 음절 단위로 태깅하는 방법들이 제안되었고, 계속해서 이를 개선하는 연구들이 진행되어 왔다.
    %음절 단위 형태소 분석은 기계적인 관점에서 음절 단위로 태깅 후 원형복원 사전을 적용하거나, 원형을 미리 복원한 상태에서 음절 단위로 태깅하는 방법이 있다.
    %그러나 음절 단위 형태소 분석은 형태소의 경계를 정확하게 파악하기 힘든 면과 시퀀스의 길이가 길어지면서 장기적인 문맥 정보를 학습하기 어려운 면도 있다.
    %본 논문에서는 전자를 사전 기반 형태소 분석이라 하고, 후자를 음절 단위 형태소 분석이라고 하겠다.
    %또한 두 방법 모두 수동으로 레이블링된 말뭉치를 기반으로 학습하기 때문에 학습말뭉치에 등장하지 않은 새로운 음절 조합이나 형태소에 대해서는 정확한 분석을 수행하지 못하는 한계가 공통적으로 있다.
    %최근에는 인터넷의 발전과 함께 오픈소스 및 오픈데이터의 확산 분위기 속에서 다양한 사람들에 의해 공유된 웹텍스트, 말뭉치, 언어자원, 지식 등이 상당히 축적해가고 있다.
    %이를 통해 사전 구축 및 유지 비용이 줄어들 수 있는 것은 사전 기반 방법의 한계를 극복할 수 있는 좋은 기회가 될 수 있다.

    Against this background, this paper considers how the dictionary-based morphological analysis method used by MeCab, an open software used for Korean and Japanese morphological analysis in tokenisers, an essential tool for preprocessing for deep learning, can be effectively improved through deep learning, and proposes a method.
    The dictionary-based morphological analysis method[Kudo2004, Na2014] trained by the CRF method[Sutton2012] lists the candidate morphemes in the dictionary from a given sentence to form a lattice structure connected by a directed graph, and finds the optimal morphological analysis path within it.
    The process of finding the best path in Lattice uses the Viterbi algorithm[Forney1973], which finds the path that minimises the cost of each morpheme node and the sum of the neighbouring costs of two consecutive morphemes.
    The main types of errors in these dictionary-based morphological analysis methods are when new words not in the dictionary are used in a sentence, or when the optimal path calculation selects the wrong result due to a bias.
    For example, it may be less costly to choose one long morpheme than to choose several short ones, but it can sometimes be the wrong analysis.
    The main motivation for this research is that the path that minimises the sum of all costs for nodes and connections may not be the optimal path.

    %이러한 배경하에서 본 논문에서는 딥러닝을 위한 전처리의 필수 도구인 토큰화기에서 한국어 및 일본어 형태소 분석을 위해 사용되는 공개 소프트웨어인 MeCab에서 사용하는 사전 기반 형태소 분석 방법을 딥러닝을 통해 효과적으로 개선할 수 있는 방법에 대해 고려하고 방법을 제안하고자 한다.
    %CRF 방법으로 학습하는 사전 기반 형태소 분석 방법은 주어진 문장으로부터 사전에 있는 선택가능한 형태소 후보들을 나열하여 방향성 있는 그래프로 연결한 Lattice 구조를 형성하고, 그 안에서 최적의 형태소 분석 경로를 찾아내는 방법이다.
    %Lattice에서 최적의 경로를 찾아내는 과정은 Viterbi 알고리즘을 사용하며, 이 때의 경로는 각 형태소 노드에 대한 비용과 연속적인 두 형태소의 인접 비용의 합이 최소가 되는 경로를 찾아내는 것이다.
    %이러한 사전 기반 형태소 분석 방법에서의 주된 오류 유형은 사전에 없는 새로운 어휘가 문장에서 사용된 경우이거나, 최적 경로 계산 시에 잘못된 편견으로 인해 오분석 결과를 선택한 경우이다.
    %예를 들면 짧은 형태소 여럿을 선택하는 것보다 하나의 긴 형태소를 선택하는 것이 비용적으로는 적을 수 있지만 때에 따라 잘못된 분석이 될 수도 있다.
    %이로 인해 노드 및 연결에 대한 모든 비용의 합이 최소가 되는 경로가 최적의 경로가 아닐 수 있다는 점이 본 연구의 주된 동기이다.

    In order to identify the cases where a suboptimal solution is actually the best solution according to the best path calculation, we modified the best path calculation method to generate suboptimal analysis results and check the extent to which they are correct.
    While there could be a variety of suboptimal choices, we used the method of replacing one morpheme node on the optimal path with nodes ranked 2 through 5.
    As shown in [Table 1], we were able to confirm the extent to which the analysis performance could be improved by replacing the optimal path with a lower-ranked node.
    We can think of the problem of finding the actual correct answer among these generated suboptimals as similar to the problem of re-ranking search results in information retrieval.
    Previously, in [Choi2018], N-Best analysis results generated by the seq2seq model were re-ranked based on a convolutional neural network to improve performance.
    In this study, reranking was performed using two BERT models of different types and forms as proposed in [Nogueira2019].
    The experimental results show that the first-stage reranking improves performance by more than 34\% from the previous written and spoken models, and the second-stage reranking with a different type of input and a different kind of pre-trained model further improves performance by more than 40\% the previous written and spoken models.

    %최적 경로 계산 상으로 차선책이 실제적인 최선책인 경우들를 파악하기 위해 본 연구에서는 최적 경로 계산 방법을 수정하여 차선책 분석 결과들을 생성해보고 이들 중에 정답이 있는 정도를 확인하였다.
    %다양한 차선책 선택 방법이 있을 수 있겠지만, 최적 경로 상에서 하나의 형태소 노드를 2~5순위의 노드로 치환하는 방법을 사용하였다.
    %[표 1]에서 보는 바와 같이, 최적 경로를 차순위 노드로 치환함으로 분석 성능을 개선할 수 있는 정도를 확인할 수 있었다.
    %이렇게 생성된 차선책들 중에서 실제 정답을 찾는 문제는 정보검색에서 검색 결과를 재순위화하는 문제와 유사한 것으로 생각해 볼 수 있다.
    %기존에는 [Choi2018] 연구에서 seq2seq 모델 생성한 N-Best 분석 결과들을 컨볼루션 신경망 기반의 재순위화를 통해 성능을 개선하였다.
    %본 연구에서는 [Nogueira2019]에서 제안한 것처럼 서로 다른 종류 및 형태로 두 개의 BERT 모델을 사용하여 재순위화를 수행하였다.
    %실험 결과를 통해, 1단계 재순위화를 통해 문어체와 구어체 모델에서 기존보다 34% 이상 오류가 감소하여 1차로 성능을 개선하였고, 다른 형태의 입력과 다른 종류의 사전학습 모델을 사용한 2단계 재순위화를 수행함으로 추가적으로 기존 보다 오류가 40% 이상 감소한 성능 개선 결과를 확인할 수 있었다.

    Through this method, the performance of the dictionary-based morphological analysis method could be further improved, but the analysis speed slows down when the morphological analysis system is configured including the re-ranking model itself.
    However, it is possible to use the results of multiple re-ranked morpheme analyses to update the connection costs between morphemes in the dictionary, similar to the backpropagation process in a typical neural network.
    It is also expected that the morphological analysis system with improved connection costs will be able to generate better reranking candidates, which will further improve performance by doing this iteratively.
    For this, additional research is needed in the future, and in this paper, only the performance improvement through the second-stage re-ranking was covered as the scope of the study.
    The main contributions of this study are as follows:
    \begin{itemize}
        \item 1. [Further improvement of dictionary-based morphological analysis method using suboptimal analysis results]: We explore the possibility of performance improvement by introducing a method to replace the optimal path with a suboptimal node, and propose a method to effectively improve the dictionary-based morphological analysis method through deep learning.
        \item 2. [Extend the performance improvement by introducing a two-stage reranking model]: To improve the performance of dictionary-based analysis by reranking the morphological analysis results, we propose to extend the performance improvement by using different BERT models to perform two rounds of reranking.
        \item 3. [A method for updating connection costs in dictionary and suggestions for future research]: We propose a new method for updating dictionary connection costs based on the results of re-ranked morphological analysis, and provide directions for future research by suggesting improvements and directions for future research.
    \end{itemize}
    These contributions provide important insights into the performance improvement of Korean morphological analysis and the direction of future research, and will serve as a useful reference for future researchers.

    %이러한 방법을 통해 사전 기반 형태소 분석 방법의 성능을 추가로 개선할 수 있었으나, 재순위화 모델 자체를 포함하여 형태소 분석 시스템을 구성하게 되면 분석 속도가 느려지는 문제가 발생한다.
    %그러나 재순위화된 여러 형태소 분석 결과들을 사용해서 일반적인 Neural Network의 Back Propagration과정처럼 사전 상의 형태소 간의 연결 비용을 업데이트하는 방식을 적용해볼 수 있다.
    %또한 연결 비용이 개선된 형태소 분석 시스템을 통해 더 좋은 재순위화 후보들을 생성할 수 있게 되므로, 이를 반복적으로 수행함으로써 성능을 더욱 개선할 수 있을 것으로도 기대된다.
    %이에 대해서는 앞으로 추가적인 연구가 필요하며, 본 논문에서는 2단계 재순위화를 통한 성능 개선에 대해서만 연구 범위로 다루었다.
    %본 연구가 가지는 연구의 주요 기여점은 다음과 같습니다:
    % 1. **차선 분석 결과를 활용한 사전 기반 형태소 분석 방법의 추가 개선**: 최적 경로를 차순위 노드로 치환하는 방법을 도입하여 성능 개선의 가능성을 탐색하고, 사전 기반 형태소 분석 방법을 딥러닝을 통해 효과적으로 개선할 수 있는 방안을 제안한다.
    % 2. **2단계 재순위화 모델 도입을 통한 성능 개선 정도 확대**: 형태소 분석 결과의 재순위화를 통해 사전 기반 분석의 성능을 개선하기 위해 서로 다른 BERT 모델을 사용하여 두 차례의 재순위화를 과정을 통해 성능 개선 정도를 확대시킬 수 있는 방안을 제안한다.
    % 3. **사전 연결 비용의 업데이트 방법 및 향후 연구 제안**: 재순위화된 형태소 분석 결과를 기반으로 사전의 연결 비용을 업데이트하는 새로운 방식을 제안하고, 향후 연구 방향과 개선점을 제시하여 이후 연구에 대한 방향성을 제공한다.
    %본 논문은 이러한 기여점들을 통해 한국어 형태소 분석의 성능 개선과 연구의 진행 방향에 중요한 통찰을 제공하며, 이후 연구자들에게 유용한 참고자료가 될 것입니다.

    This paper is organised as follows:
    In section 2, we introduce previous research cases related to this study, and in section 3, we discuss how to configure and train a dictionary-based morphological analysis system.
    In section 4, we discuss how to generate secondary results of morphological analysis and create re-ranking data, and propose a method to train a two-stage re-ranking model.
    In section 5, we discuss the results of performance improvement through morphological analysis model and re-ranking model, and in section 6, we discuss error analysis and performance improvement cases.
    Finally, in section 7, we conclude the paper and discuss the limitations of this study and the direction of future research.

    %본 논문의 구성은 다음과 같다.
    %2장에서 본 연구와 관련한 이전 연구 사례들을 소개하며, 3장에서는 사전 기반 형태소 분석 시스템을 구성하고 학습하는 방법에 대해 다룬다.
    %4장에서는 학습된 형태소 분석 모델의 차선 결과들을 생성하여 재순위화 데이터를 만드는 방법에 대해 다루고, 2단계 재순위화 모델을 학습시키는 방법을 제안한다.
    %5장에서는 형태소 분석 모델 및 재순위화 모델을 통한 성능 개선 실험 결과를 다루고, 6장에서 오류 분석 및 성능 개선 사례들을 살펴본 뒤, 7장에서 본 연구의 결론과 향후 연구를 제시한다.


    \section{Sample for first level head}\label{secX}

    Lorem ipsum dolor sit amet, consectetuer adipiscing elit~\cite{Rothermel1997} Ut purus elit, vestibulum ut, placerat ac, adipiscing vitae,
    felis. Curabitur dictum gravida mauris. Nam arcu libero, nonummy eget, consectetuer id, vulputate a, magna. Donec
    vehicula augue eu neque. Pellentesque habitant morbi tristique senectus et netus et malesuada fames ac turpis egestas.
    Mauris ut leo. Cras viverra metus rhoncus sem. Nulla et lectus vestibulum urna fringilla ultrices. Phasellus eu tellus
    sit amet tortor gravida placerat. Integer sapien est, iaculis in, pretium quis, viverra ac, nunc. Praesent eget sem vel leo ultrices bibendum. Aenean faucibus. Morbi dolor nulla, malesuada eu, pulvinar at, mollis ac, nulla. Curabitur
    auctor semper nulla. Donec varius orci eget risus. Duis nibh mi, congue eu, accumsan eleifend, sagittis quis, diam.
    Duis eget orci sit amet orci dignissim rutrum.
    \begin{eqnarray}
        s(nT_{s}) &= &s(t)\times \sum\limits_{n=0}^{N-1} \delta (t-nT_{s}) \xleftrightarrow{\mathrm{DFT}}  S \left(\frac{m}{NT_{s}}\right) \nonumber\\
        &= &\frac{1}{N} \sum\limits_{n=0}^{N-1} \sum\limits_{k=-N/2}^{N/2-1} s_{k} e^{\mathrm{j}2\pi k\Delta fnT_{s}} e^{-j\frac{2\pi}{N}mn}
    \end{eqnarray}


    \section{Sample for another first level head \cite{Rothermel1998}}\label{sec2}

    Nulla malesuada porttitor diam. Donec felis erat, congue non, volutpat at, tincidunt tristique, libero. Vivamus viverra
    fermentum felis. Donec nonummy pellentesque ante. Phasellus adipiscing semper elit. Proin fermentum massa ac quam. Sed diam turpis, molestie vitae, placerat a, molestie nec, leo \cite{Elbaum2002} Maecenas lacinia. Nam ipsum ligula, eleifend at, accumsan nec, suscipit a, ipsum. Morbi blandit ligula feugiat magna. Nunc eleifend consequat lorem. Sed lacinia nulla vitae enim. Pellentesque tincidunt purus vel magna. Integer non enim. Praesent euismod nunc eu purus. Donec
    bibendum quam in tellus. Nullam cursus pulvinar lectus. Donec et mi. Nam vulputate metus eu enim. Vestibulum
    pellentesque felis eu massa.

    Example for bibliography citations cite~\citep{Elbaum2002}, cites~\cite{Allen2011,Yoo2007}

    Quisque ullamcorper placerat ipsum. Cras nibh~\cite{Yoo2007,Schulz2012} Morbi vel justo vitae lacus tincidunt ultrices. Lorem ipsum dolor sit
    amet, consectetuer adipiscing elit. In hac habitasse platea dictumst. Integer tempus convallis augue. Etiam facilisis.
    Nunc elementum fermentum wisi. Aenean placerat. Ut imperdiet, enim sed gravida sollicitudin, felis odio placerat quam, ac pulvinar elit purus eget enim. Nunc vitae tortor. Proin tempus nibh sit amet nisl. Vivamus quis tortor
    vitae risus porta vehicula.

    Fusce mauris. Vestibulum luctus nibh at lectus. Sed bibendum, nulla a faucibus semper, leo velit ultricies tellus, ac venenatis arcu wisi vel nisl. Vestibulum diam. Aliquam pellentesque, augue quis sagittis posuere, turpis lacus congue quam, in hendrerit risus eros eget felis. Maecenas eget erat in sapien mattis porttitor. Vestibulum porttitor. Nulla facilisi. Sed a turpis eu lacus commodo facilisis. Morbi fringilla, wisi in dignissim interdum, justo lectus sagittis dui, et vehicula libero dui cursus dui. Mauris tempor ligula sed lacus. Duis cursus enim ut augue. Cras ac magna. Cras nulla.
    Nulla egestas. Curabitur a leo. Quisque egestas wisi eget nunc. Nam feugiat lacus vel est. Curabitur consectetuer.


    \begin{figure*}[t]
        \centerline{\includegraphics[width=342pt,height=9pc,draft]{empty}}
        \caption{This is the sample figure caption.\label{fig1}}
    \end{figure*}

    Suspendisse vel felis. Ut lorem lorem, interdum eu, tincidunt sit amet, laoreet vitae, arcu. Aenean faucibus pede eu ante. Praesent enim elit, rutrum at, molestie non, nonummy vel, nisl. Ut lectus eros, malesuada sit amet, fermentum
    eu, sodales cursus, magna. Donec eu purus. Quisque vehicula, urna sed ultricies auctor, pede lorem egestas dui, and convallis elit erat sed nulla. Donec luctus. Curabitur et nunc. Aliquam dolor odio, commodo pretium, ultricies non,
    pharetra in, velit. Integer arcu est, nonummy in, fermentum faucibus, egestas vel, odio.

    Sed commodo posuere pede. Mauris ut est. Ut quis purus. Sed ac odio. Sed vehicula hendrerit sem. Duis non
    odio. Morbi ut dui. Sed accumsan risus eget odio. In hac habitasse platea dictumst. Pellentesque non-elit. Fusce
    sed justo eu urna porta tincidunt. Mauris felis odio, sollicitudin sed, volutpat a, ornare ac, erat. Morbi quis dolor.
    Donec pellentesque, erat ac sagittis semper, nunc dui lobortis purus, quis congue purus metus ultricies tellus. Proin et quam. Class aptent taciti sociosqu ad litora torquent per conubia nostra, per inceptos hymenaeos. Praesent sapien turpis, fermentum vel, eleifend faucibus, vehicula eu, lacus.

    \begin{figure*}
        \centerline{\includegraphics[width=342pt,height=9pc,draft]{empty}}
        \caption{This is the sample figure caption.\label{fig2}}
    \end{figure*}

    \subsection{Example for second level head}

    Pellentesque habitant morbi tristique senectus et netus et malesuada fames ac turpis egestas. Donec odio elit, dictum
    in, hendrerit sit amet, egestas sed, leo. Praesent feugiat sapien aliquet odio. Integer vitae justo. Aliquam vestibulum fringilla lorem. Sed neque lectus, consectetuer at, consectetuer sed, eleifend ac, lectus. Nulla facilisi. Pellentesque
    eget lectus. Proin eu metus. Sed porttitor. In hac habitasse platea dictumst. Suspendisse eu lectus. Ut mi mi, lacinia
    sit amet, placerat et, mollis vitae, dui. Sed ante tellus, tristique ut, iaculis eu, malesuada ac, dui. Mauris nibh leo,
    facilisis non, adipiscing quis, ultrices a, dui.

    Morbi luctus, wisi viverra faucibus pretium, nibh est placerat odio, nec commodo wisi enim eget quam. Quisque libero justo, consectetuer a, feugiat vitae, porttitor eu, libero. Suspendisse sed mauris vitae elit sollicitudin malesuada.

    Maecenas ultricies eros sit amet ante. Ut venenatis velit. Maecenas sed mi eget dui varius euismod. Phasellus aliquet
    volutpat odio. Vestibulum ante ipsum primis in faucibus orci luctus et ultrices posuere cubilia Curae; Pellentesque sit amet pede ac sem eleifend consectetuer. Nullam elementum, urna vel imperdiet sodales, elit ipsum pharetra ligula,
    ac pretium ante justo a nulla. Curabitur tristique arcu eu metus. Vestibulum lectus. Proin mauris. Proin eu nunc eu urna hendrerit faucibus. Aliquam auctor, pede consequat laoreet varius, eros tellus scelerisque quam, pellentesque hendrerit ipsum dolor sed augue. Nulla nec lacus.

    \begin{quote}
        This is an example~\citep{Elbaum2002,Blanchard2015} for quote text. This is an example for quote text. This is an example for quote text. This is an example for quote text~\cite{Blanchard2015}. This is an example for quote text. This is an example for quote text. This is an example for quote text. This is an example for quote text. This is an example for quote text. This is an example for quote text~\cite{Blanchard2015}. This is an example for quote text. This is an example for quote text. This is an example for quote text.
    \end{quote}


    \section{Sample for next first level head}\label{sec3}

    \subsection{Example for another second level head}

    Suspendisse vitae elit. Aliquam arcu neque, ornare in, ullamcorper quis, commodo eu, libero. Fusce sagittis erat at
    erat tristique mollis. Maecenas sapien libero, molestie et, lobortis in, sodales eget, dui. Morbi ultrices rutrum lorem.
    Nam elementum ullamcof aaer leo. Morbi dui. Aliquam sagittis. Nunc placerat. Pellentesque tristique sodales est.
    Maecenas imperdiet lacinia velit. Cras non urna. Morbi eros pede, suscipit ac, varius vel, egestas non, eros. Praesent
    malesuada, diam id pretium elementum, eros sem dictum tortor, vel consectetuer odio sem sed wisi.

    Sed feugiat. Cum sociis natoque penatibus et magnis dis parturient montes, nascetur ridiculus mus. Ut pellentesque
    augue sed urna. Vestibulum diam eros, fringilla et, consectetuer eu, nonummy id, sapien. Nullam at lectus. In sagittis
    ultrices mauris. Curabitur malesuada erat sit amet massa. Fusce blandit. Aliquam erat volutpat. Aliquam euismod.
    Aenean vel lectus. Nunc imperdiet justo nec dolor.

    \subsection{Second level head text}

    Etiam euismod. Fusce facilisis lacinia dui. Suspendisse potenti. In mi erat, cursus id, nonummy sed, ullamcorper
    eget, sapien. Praesent pretium, magna in eleifend egestas, pede pede pretium lorem, quis consectetuer tortor sapien
    facilisis magna. Mauris quis magna varius nulla scelerisque imperdiet. Aliquam non quam. Aliquam porttitor quam
    a lacus. Praesent vel arcu ut tortor cursus volutpat. In vitae pede quis diam bibendum placerat. Fusce elementum
    convallis neque. Sed dolor orci, scelerisque ac, dapibus nec, ultricies ut, mi. Duis nec dui quis leo sagittis commodo.

    \subsubsection{Third level head text}

    Aliquam lectus. Vivamus leo. Quisque ornare tellus ullamcorper nulla. Mauris porttitor pharetra tortor. Sed fringilla
    justo sed mauris. Mauris tellus. Sed non leo. Nullam elementum, magna in cursus sodales, augue est scelerisque
    sapien, venenatis congue nulla arcu et pede. Ut suscipit enim vel sapien. Donec congue. Maecenas urna mi, suscipit
    in, placerat ut, vestibulum ut, massa. Fusce ultrices nulla et nisl.

    Etiam ac leo a risus tristique nonummy. Donec dignissim tincidunt nulla. Vestibulum rhoncus molestie odio. Sed
    lobortis, justo et pretium lobortis, mauris turpis condimentum augue, nec ultricies nibh arcu pretium enim. Nunc
    purus neque, placerat id, imperdiet sed, pellentesque nec, nisl. Vestibulum imperdiet neque non sem accumsan laoreet.
    In hac habitasse platea dictumst. Etiam condimentum facilisis libero. Suspendisse in elit quis nisl aliquam dapibus.
    Pellentesque auctor sapien. Sed egestas sapien nec lectus. Pellentesque vel dui vel neque bibendum viverra. Aliquam
    porttitor nisl nec pede. Proin mattis libero vel turpis. Donec rutrum mauris et libero. Proin euismod porta felis.
    Nam lobortis, metus quis elementum commodo, nunc lectus elementum mauris, eget vulputate ligula tellus eu neque.
    Vivamus eu dolor.

    Nulla in ipsum. Praesent eros nulla, congue vitae, euismod ut, commodo a, wisi. Pellentesque habitant morbi
    tristique senectus et netus et malesuada fames ac turpis egestas. Aenean nonummy magna non leo. Sed felis erat,
    ullamcorper in, dictum non, ultricies ut, lectus. Proin vel arcu a odio lobortis euismod. Vestibulum ante ipsum primis
    in faucibus orci luctus et ultrices posuere cubilia Curae; Proin ut est. Aliquam odio. Pellentesque massa turpis, cursus
    eu, euismod nec, tempor congue, nulla. Duis viverra gravida mauris. Cras tincidunt. Curabitur eros ligula, varius ut,
    pulvinar in, cursus faucibus, augue.



    \begin{boxtext}
        \noindent{\bf Example of Boxtext\ }\\
        This is sample for boxtext this is sample for boxtext this is sample for boxtext this is sample for boxtext this is sample for boxtext this is sample for boxtext this is sample for boxtext this is sample for boxtext this is sample for boxtext this is sample for boxtext this is sample for boxtext this is sample for boxtext this is sample for boxtext this is sample for boxtext this is sample for boxtext this is sample for boxtext this is sample for boxtext this is sample for boxtext this is sample for boxtext this is sample for boxtext this is sample for boxtext this is sample for boxtext this is sample for boxtext this is sample for boxtext this is sample for boxtext this is sample for boxtext this is sample for boxtext this is sample for boxtext this is sample for boxtext this is sample for boxtext this is sample for boxtext this is sample for boxtext this is sample for boxtext this is sample for boxtext this is sample for boxtext this is sample for boxtext this is sample for boxtext this is sample for boxtext this is sample for boxtext this is sample for boxtext this is sample for boxtext this is sample for boxtext this is sample for boxtext this is sample for boxtext this is sample for boxtext this is sample for boxtext this is sample for boxtext.
    \end{boxtext}

    \paragraph{Fourth level head text}

    Sed feugiat. Cum sociis natoque penatibus et magnis dis parturient montes, nascetur ridiculus mus. Ut pellentesque
    augue sed urna. Vestibulum diam eros, fringilla et, consectetuer eu, nonummy id, sapien. Nullam at lectus. In sagittis
    ultrices mauris. Curabitur malesuada erat sit amet massa. Fusce blandit. Aliquam erat volutpat. Aliquam euismod.
    Aenean vel lectus. Nunc imperdiet justo nec dolor.

    Etiam euismod. Fusce facilisis lacinia dui. Suspendisse potenti. In mi erat, cursus id, nonummy sed, ullamcorper
    eget, sapien. Praesent pretium, magna in eleifend egestas, pede pede pretium lorem, quis consectetuer tortor sapien
    facilisis magna. Mauris quis magna varius nulla scelerisque imperdiet. Aliquam non quam. Aliquam porttitor quam
    a lacus. Praesent vel arcu ut tortor cursus volutpat. In vitae pede quis diam bibendum placerat. Fusce elementum
    convallis neque. Sed dolor orci, scelerisque ac, dapibus nec, ultricies ut, mi. Duis nec dui quis leo sagittis commodo.

    \subparagraph{Fifth level head text}

    Aliquam lectus. Vivamus leo. Quisque ornare tellus ullamcorper nulla. Mauris porttitor pharetra
    tortor. Sed fringilla justo sed mauris. Mauris tellus. Sed non leo. Nullam elementum, magna in cursus sodales, augue
    est scelerisque sapien, venenatis congue nulla arcu et pede. Ut suscipit enim vel sapien. Donec congue. Maecenas
    urna mi, suscipit in, placerat ut, vestibulum ut, massa. Fusce ultrices nulla et nisl.

    Etiam ac leo a risus tristique nonummy. Donec dignissim tincidunt nulla. Vestibulum rhoncus molestie odio. Sed
    lobortis, justo et pretium lobortis, mauris turpis condimentum augue, nec ultricies nibh arcu pretium enim. Nunc
    purus neque, placerat id, imperdiet sed, pellentesque nec, nisl. Vestibulum imperdiet neque non sem accumsan laoreet.
    In hac habitasse platea dictumst. Etiam condimentum facilisis libero. Suspendisse in elit quis nisl aliquam dapibus.
    Pellentesque auctor sapien. Sed egestas sapien nec lectus. Pellentesque vel dui vel neque bibendum viverra. Aliquam
    porttitor nisl nec pede. Proin mattis libero vel turpis. Donec rutrum mauris et libero. Proin euismod porta felis.
    Nam lobortis, metus quis elementum commodo, nunc lectus elementum mauris, eget vulputate ligula tellus eu neque.
    Vivamus eu dolor.

    in faucibus orci luctus et ultrices posuere cubilia Curae; Proin ut est. Aliquam odio. Pellentesque massa turpis, cursus
    eu, euismod nec, tempor congue, nulla. Duis viverra gravida mauris. Cras tincidunt. Curabitur eros ligula, varius ut,
    pulvinar in, cursus faucibus, augue.

    Curabitur tellus magna, porttitor a, commodo a, commodo in, tortor. Donec interdum. Praesent scelerisque. Mae-
    cenas posuere sodales odio. Vivamus metus lacus, varius quis, imperdiet quis, rhoncus a, turpis. Etiam ligula arcu,
    elementum a, venenatis quis, sollicitudin sed, metus. Donec nunc pede, tincidunt in, venenatis vitae, faucibus vel,
    nibh. Pellentesque wisi. Nullam malesuada. Morbi ut tellus ut pede tincidunt porta. Lorem ipsum dolor sit amet,
    consectetuer adipiscing elit. Etiam congue neque id dolor.

    Donec et nisl at wisi luctus bibendum. Nam interdum tellus ac libero. Sed sem justo, laoreet vitae, fringilla at,
    adipiscing ut, nibh. Maecenas non sem quis tortor eleifend fermentum. Etiam id tortor ac mauris porta vulputate.
    Integer porta neque vitae massa. Maecenas tempus libero a libero posuere dictum. Vestibulum ante ipsum primis in
    faucibus orci luctus et ultrices posuere cubilia Curae; Aenean quis mauris sed elit commodo placerat. Class aptent
    taciti sociosqu ad litora torquent per conubia nostra, per inceptos hymenaeos. Vivamus rhoncus tincidunt libero.
    Etiam elementum pretium justo. Vivamus est. Morbi a tellus eget pede tristique commodo. Nulla nisl. Vestibulum
    sed nisl eu sapien cursus rutrum.

    Nulla non mauris vitae wisi posuere convallis. Sed eu nulla nec eros scelerisque pharetra. Nullam varius. Etiam
    dignissim elementum metus. Vestibulum faucibus, metus sit amet mattis rhoncus, sapien dui laoreet odio, nec ultricies
    nibh augue a enim. Fusce in ligula. Quisque at magna et nulla commodo consequat. Proin accumsan imperdiet sem.
    Nunc porta. Donec feugiat mi at justo. Phasellus facilisis ipsum quis ante. In ac elit eget ipsum pharetra faucibus.
    Maecenas viverra nulla in massa.

    Nulla in ipsum. Praesent eros nulla, congue vitae, euismod ut, commodo a, wisi. Pellentesque habitant morbi
    tristique senectus et netus et malesuada fames ac turpis egestas. Aenean nonummy magna non leo. Sed felis erat,
    ullamcorper in, dictum non, ultricies ut, lectus. Proin vel arcu a odio lobortis euismod. Vestibulum ante ipsum primis


    \begin{center}
        \begin{table*}[t]%
            \caption{This is sample table caption.\label{tab1}}
            \centering
            \begin{tabular*}{500pt}{@{\extracolsep\fill}lccD{.}{.}{3}c@{\extracolsep\fill}}
                \toprule
                & \multicolumn{2}{@{}c@{}}{\textbf{Spanned heading\tnote{1}}} & \multicolumn{2}{@{}c@{}}{\textbf{Spanned heading\tnote{2}}} \\\cmidrule{2-3}\cmidrule{4-5}
                \textbf{col1 head} & \textbf{col2 head} & \textbf{col3 head} & \multicolumn{1}{@{}l@{}}{\textbf{col4 head}}  & \textbf{col5 head}   \\
                \midrule
                col1 text          & col2 text          & col3 text          & 12.34                                        & col5 text\tnote{1} \\
                col1 text          & col2 text          & col3 text          & 1.62                                         & col5 text\tnote{2} \\
                col1 text          & col2 text          & col3 text          & 51.809                                       & col5 text          \\
                \bottomrule
            \end{tabular*}
            \begin{tablenotes}%%[341pt]
                \item Source: Example for table source text.
                \item[1] Example for a first table footnote.
                \item[2] Example for a second table footnote.
            \end{tablenotes}
        \end{table*}
    \end{center}

    Fusce mauris. Vestibulum luctus nibh at lectus. Sed bibendum, nulla a faucibus semper, leo velit ultricies tellus, ac
    venenatis arcu wisi vel nisl. Vestibulum diam. Aliquam pellentesque, augue quis sagittis posuere, turpis lacus congue
    quam, in hendrerit risus eros eget felis. Maecenas eget erat in sapien mattis porttitor. Vestibulum porttitor. Nulla
    facilisi. Sed a turpis eu lacus commodo facilisis. Morbi fringilla, wisi in dignissim interdum, justo lectus sagittis dui, et
    vehicula libero dui cursus dui. Mauris tempor ligula sed lacus. Duis cursus enim ut augue. Cras ac magna. Cras nulla.
    Nulla egestas. Curabitur a leo. Quisque egestas wisi eget nunc. Nam feugiat lacus vel est. Curabitur consectetuer.



    \begin{center}
        \begin{table*}[t]%
            \centering
            \caption{This is sample table caption.\label{tab2}}%
            \begin{tabular*}{500pt}{@{\extracolsep\fill}lcccc@{\extracolsep\fill}}
                \toprule
                \textbf{col1 head} & \textbf{col2 head} & \textbf{col3 head} & \textbf{col4 head} & \textbf{col5 head}          \\
                \midrule
                col1 text          & col2 text          & col3 text          & col4 text          & col5 text\tnote{$\dagger$}  \\
                col1 text          & col2 text          & col3 text          & col4 text          & col5 text                   \\
                col1 text          & col2 text          & col3 text          & col4 text          & col5 text\tnote{$\ddagger$} \\
                \bottomrule
            \end{tabular*}
            \begin{tablenotes}
                \item Source: Example for table source text.
                \item[$\dagger$] Example for a first table footnote.
                \item[$\ddagger$] Example for a second table footnote.
            \end{tablenotes}
        \end{table*}
    \end{center}



    Below is the example~\cite{Rothermel1998,Yoo2007,Schulz2012} for bulleted list. Below is the example for bulleted list. Below is the example for bulleted list. Below is the example for bulleted list. Below is the example for bulleted list. Below is the example for bulleted list\footnote{This is an example for footnote.}:
    \begin{itemize}
        \item bulleted list entry sample bulleted list entry~\cite{Allen2011} sample list entry text.
        \item bulleted list entry sample bulleted list entry. bulleted list entry sample bulleted list entry. bulleted list entry sample bulleted list entry.
        \item bulleted list entry sample bulleted list entry~\cite{Ballen2011} bulleted list entry sample bulleted list entry~\citet{Allen2011} sample list entry text. bulleted list entry sample bulleted list entry.
        \item sample list entry text. sample list entry text.
    \end{itemize}

    Suspendisse vel felis. Ut lorem lorem, interdum eu, tincidunt sit amet, laoreet vitae, arcu. Aenean faucibus pede eu
    ante. Praesent enim elit, rutrum at, molestie non, nonummy vel, nisl. Ut lectus eros, malesuada sit amet, fermentum
    eu, sodales cursus, magna. Donec eu purus. Quisque vehicula, urna sed ultricies auctor, pede lorem egestas dui, et
    convallis elit erat sed nulla. Donec luctus. Curabitur et nunc. Aliquam dolor odio, commodo pretium, ultricies non,
    pharetra in, velit. Integer arcu est, nonummy in, fermentum faucibus, egestas vel, odio.

    Sed commodo posuere pede. Mauris ut est. Ut quis purus. Sed ac odio. Sed vehicula hendrerit sem. Duis non
    odio. Morbi ut dui. Sed accumsan risus eget odio. In hac habitasse platea dictumst. Pellentesque non elit. Fusce
    sed justo eu urna porta tincidunt. Mauris felis odio, sollicitudin sed, volutpat a, ornare ac, erat. Morbi quis dolor. Donec pellentesque, erat ac sagittis semper, nunc dui lobortis purus, quis congue purus metus ultricies tellus. Proin
    et quam. Class aptent taciti sociosqu ad litora torquent per conubia nostra, per inceptos hymenaeos. Praesent sapien
    turpis, fermentum vel, eleifend faucibus, vehicula eu, lacus.


    Below is the sample for description list. Below is the example for description list. Below is the example for description list. Below is the example for description list. Below is the example for description list. Below is the example for description list:\\[12pt]


    \noindent\textbf{Description sample:}

    \begin{description}
        \item[first entry] description text. description text. description text. description text. description text. description text. description text.
        \item[second long entry] description text. description text. description text. description text. description text. description text. description text.
        \item[third entry] description text. description text. description text. description text. description text.
        \item[fourth entry] description text. description text.
    \end{description}


    \noindent\textbf{Numbered list items sample:}

    \begin{enumerate}[1.]
        \item First level numbered list entry. sample numbered list entry.

        \item First numbered list entry. sample numbered list entry. Numbered list entry. sample numbered list entry. Numbered list entry. sample numbered list entry.

        \begin{enumerate}[a.]
            \item Second level alpabetical list entry. Second level alpabetical list entry. Second level alpabetical list entry~\citet{Allen2011} Second level alpabetical list entry.

            \item Second level alpabetical list entry. Second level alpabetical list entry~\citet{Schulz2012,Allen2011,Ballen2011}.

            \begin{enumerate}[ii.]
                \item Third level lowercase roman numeral list entry. Third level lowercase roman numeral list entry. Third level lowercase roman numeral list entry.

                \item Third level lowercase roman numeral list entry. Third level lowercase roman numeral list entry~\cite{Yoo2007}.
            \end{enumerate}

            \item Second level alpabetical list entry. Second level alpabetical list entry~\cite{Elbaum2002}.
        \end{enumerate}

        \item First level numbered list entry. sample numbered list entry. Numbered list entry. sample numbered list entry. Numbered list entry.

        \item Another first level numbered list entry. sample numbered list entry. Numbered list entry. sample numbered list entry. Numbered list entry.
    \end{enumerate}

    \noindent\textbf{un-numbered list items sample:}

    \begin{enumerate}[]
        \item Sample unnumberd list text..
        \item Sample unnumberd list text.
        \item sample unnumberd list text.
        \item Sample unnumberd list text.
    \end{enumerate}


    \section{Examples for enunciations}\label{sec4}

    \begin{theorem}[Theorem subhead]
        \label{thm1}
        Example theorem text. Example theorem text. Example theorem text. Example theorem text. Example theorem text. Example theorem text. Example theorem text. Example theorem text. Example theorem text. Example theorem text. Example theorem text. Example theorem text. Example theorem text. Example theorem text. Example theorem text. Example theorem text. Example theorem text. Example theorem text. Example theorem text. Example theorem text. Example theorem text. Example theorem text. Example theorem text. Example theorem text. Example theorem text.
    \end{theorem}

    Quisque ullamcorper placerat ipsum. Cras nibh. Morbi vel justo vitae lacus tincidunt ultrices. Lorem ipsum dolor sit
    amet, consectetuer adipiscing elit. In hac habitasse platea dictumst. Integer tempus convallis augue. Etiam facilisis.
    Nunc elementum fermentum wisi. Aenean placerat. Ut imperdiet, enim sed gravida sollicitudin, felis odio placerat
    quam, ac pulvinar elit purus eget enim. Nunc vitae tortor. Proin tempus nibh sit amet nisl. Vivamus quis tortor
    vitae risus porta vehicula.

    Fusce mauris. Vestibulum luctus nibh at lectus. Sed bibendum, nulla a faucibus semper, leo velit ultricies tellus, ac
    venenatis arcu wisi vel nisl. Vestibulum diam. Aliquam pellentesque, augue quis sagittis posuere, turpis lacus congue
    quam, in hendrerit risus eros eget felis. Maecenas eget erat in sapien mattis porttitor. Vestibulum porttitor. Nulla
    facilisi. Sed a turpis eu lacus commodo facilisis. Morbi fringilla, wisi in dignissim interdum, justo lectus sagittis dui, et
    vehicula libero dui cursus dui. Mauris tempor ligula sed lacus. Duis cursus enim ut augue. Cras ac magna. Cras nulla.
    Nulla egestas. Curabitur a leo. Quisque egestas wisi eget nunc. Nam feugiat lacus vel est. Curabitur consectetuer.

    \begin{proposition}
        Example proposition text. Example proposition text. Example proposition text. Example proposition text. Example proposition text. Example proposition text. Example proposition text. Example proposition text. Example proposition text. Example proposition text. Example proposition text. Example proposition text. Example proposition text. Example proposition text. Example proposition text. Example proposition text.
    \end{proposition}

    Nulla malesuada porttitor diam. Donec felis erat, congue non, volutpat at, tincidunt tristique, libero. Vivamus
    viverra fermentum felis. Donec nonummy pellentesque ante. Phasellus adipiscing semper elit. Proin fermentum massa
    ac quam. Sed diam turpis, molestie vitae, placerat a, molestie nec, leo. Maecenas lacinia. Nam ipsum ligula, eleifend
    at, accumsan nec, suscipit a, ipsum. Morbi blandit ligula feugiat magna. Nunc eleifend consequat lorem. Sed lacinia
    nulla vitae enim. Pellentesque tincidunt purus vel magna. Integer non enim. Praesent euismod nunc eu purus. Donec
    bibendum quam in tellus. Nullam cursus pulvinar lectus. Donec et mi. Nam vulputate metus eu enim. Vestibulum
    pellentesque felis eu massa.

    Quisque ullamcorper placerat ipsum. Cras nibh. Morbi vel justo vitae lacus tincidunt ultrices. Lorem ipsum dolor sit
    amet, consectetuer adipiscing elit. In hac habitasse platea dictumst. Integer tempus convallis augue. Etiam facilisis.
    Nunc elementum fermentum wisi. Aenean placerat. Ut imperdiet, enim sed gravida sollicitudin, felis odio placerat
    quam, ac pulvinar elit purus eget enim. Nunc vitae tortor. Proin tempus nibh sit amet nisl. Vivamus quis tortor
    vitae risus porta vehicula.

    \begin{definition}[Definition sub head]
        Example definition text. Example definition text. Example definition text. Example definition text. Example definition text. Example definition text. Example definition text. Example definition text. Example definition text. Example definition text. Example definition text.
    \end{definition}

    Sed commodo posuere pede. Mauris ut est. Ut quis purus. Sed ac odio. Sed vehicula hendrerit sem. Duis non
    odio. Morbi ut dui. Sed accumsan risus eget odio. In hac habitasse platea dictumst. Pellentesque non elit. Fusce
    sed justo eu urna porta tincidunt. Mauris felis odio, sollicitudin sed, volutpat a, ornare ac, erat. Morbi quis dolor.
    Donec pellentesque, erat ac sagittis semper, nunc dui lobortis purus, quis congue purus metus ultricies tellus. Proin
    et quam. Class aptent taciti sociosqu ad litora torquent per conubia nostra, per inceptos hymenaeos. Praesent sapien
    turpis, fermentum vel, eleifend faucibus, vehicula eu, lacus.

    Pellentesque habitant morbi tristique senectus et netus et malesuada fames ac turpis egestas. Donec odio elit,
    dictum in, hendrerit sit amet, egestas sed, leo. Praesent feugiat sapien aliquet odio. Integer vitae justo. Aliquam
    vestibulum fringilla lorem. Sed neque lectus, consectetuer at, consectetuer sed, eleifend ac, lectus. Nulla facilisi.
    Pellentesque eget lectus. Proin eu metus. Sed porttitor. In hac habitasse platea dictumst. Suspendisse eu lectus. Ut
    mi mi, lacinia sit amet, placerat et, mollis vitae, dui. Sed ante tellus, tristique ut, iaculis eu, malesuada ac, dui.
    Mauris nibh leo, facilisis non, adipiscing quis, ultrices a, dui.

    \begin{proof}
        Example for proof text. Example for proof text. Example for proof text. Example for proof text. Example for proof text. Example for proof text. Example for proof text. Example for proof text. Example for proof text. Example for proof text.
    \end{proof}

    Nam dui ligula, fringilla a, euismod sodales, sollicitudin vel, wisi. Morbi auctor lorem non justo. Nam lacus libero,
    pretium at, lobortis vitae, ultricies et, tellus. Donec aliquet, tortor sed accumsan bibendum, erat ligula aliquet magna,
    vitae ornare odio metus a mi. Morbi ac orci et nisl hendrerit mollis. Suspendisse ut massa. Cras nec ante. Pellentesque
    a nulla. Cum sociis natoque penatibus et magnis dis parturient montes, nascetur ridiculus mus. Aliquam tincidunt
    urna. Nulla ullamcorper vestibulum turpis. Pellentesque cursus luctus mauris.

    Nulla malesuada porttitor diam. Donec felis erat, congue non, volutpat at, tincidunt tristique, libero. Vivamus
    viverra fermentum felis. Donec nonummy pellentesque ante. Phasellus adipiscing semper elit. Proin fermentum massa
    ac quam. Sed diam turpis, molestie vitae, placerat a, molestie nec, leo. Maecenas lacinia. Nam ipsum ligula, eleifend
    at, accumsan nec, suscipit a, ipsum. Morbi blandit ligula feugiat magna. Nunc eleifend consequat lorem. Sed lacinia
    nulla vitae enim. Pellentesque tincidunt purus vel magna. Integer non enim. Praesent euismod nunc eu purus. Donec
    bibendum quam in tellus. Nullam cursus pulvinar lectus. Donec et mi. Nam vulputate metus eu enim. Vestibulum
    pellentesque felis eu massa.

    \begin{proof}[Proof of Theorem~\ref{thm1}]
        Example for proof text. Example for proof text. Example for proof text. Example for proof text. Example for proof text. Example for proof text. Example for proof text. Example for proof text. Example for proof text. Example for proof text.
    \end{proof}

    Etiam euismod. Fusce facilisis lacinia dui. Suspendisse potenti. In mi erat, cursus id, nonummy sed, ullamcorper
    eget, sapien. Praesent pretium, magna in eleifend egestas, pede pede pretium lorem, quis consectetuer tortor sapien
    facilisis magna. Mauris quis magna varius nulla scelerisque imperdiet. Aliquam non quam. Aliquam porttitor quam
    a lacus. Praesent vel arcu ut tortor cursus volutpat. In vitae pede quis diam bibendum placerat. Fusce elementum
    convallis neque. Sed dolor orci, scelerisque ac, dapibus nec, ultricies ut, mi. Duis nec dui quis leo sagittis commodo.
    Aliquam lectus. Vivamus leo. Quisque ornare tellus ullamcorper nulla. Mauris porttitor pharetra tortor. Sed fringilla
    justo sed mauris. Mauris tellus. Sed non leo. Nullam elementum, magna in cursus sodales, augue est scelerisque
    sapien, venenatis congue nulla arcu et pede. Ut suscipit enim vel sapien. Donec congue. Maecenas urna mi, suscipit
    in, placerat ut, vestibulum ut, massa. Fusce ultrices nulla et nisl.

    Pellentesque habitant morbi tristique senectus et netus et malesuada fames ac turpis egestas. Donec odio elit,
    dictum in, hendrerit sit amet, egestas sed, leo. Praesent feugiat sapien aliquet odio. Integer vitae justo. Aliquam
    vestibulum fringilla lorem. Sed neque lectus, consectetuer at, consectetuer sed, eleifend ac, lectus. Nulla facilisi.
    Pellentesque eget lectus. Proin eu metus. Sed porttitor. In hac habitasse platea dictumst. Suspendisse eu lectus. Ut Curabitur tellus magna, porttitor a, commodo a, commodo in, tortor. Donec interdum. Praesent scelerisque. Mae-
    cenas posuere sodales odio. Vivamus metus lacus, varius quis, imperdiet quis, rhoncus a, turpis. Etiam ligula arcu,
    elementum a, venenatis quis, sollicitudin sed, metus. Donec nunc pede, tincidunt in, venenatis vitae, faucibus vel,


    \begin{sidewaystable}%[h]
        \caption{Sideways table caption. For decimal alignment refer column 4 to 9 in tabular* preamble.\label{tab3}}%
        \begin{tabular*}{\textheight}{@{\extracolsep\fill}lccD{.}{.}{4}D{.}{.}{4}D{.}{.}{4}D{.}{.}{4}D{.}{.}{4}D{.}{.}{4}@{\extracolsep\fill}}%
            \toprule
            & \textbf{col2 head} & \textbf{col3 head} & \multicolumn{1}{c}{\textbf{10}} & \multicolumn{1}{c}{\textbf{20}} & \multicolumn{1}{c}{\textbf{30}} & \multicolumn{1}{c}{\textbf{10}} & \multicolumn{1}{c}{\textbf{20}} & \multicolumn{1}{c}{\textbf{30}} \\
            \midrule
            & col2 text          & col3 text          & 0.7568                          & 1.0530                          & 1.2642                          & 0.9919                          & 1.3541                          & 1.6108                          \\
            &                    & col2 text          & 12.5701                         & 19.6603                         & 25.6809                         & 18.0689                         & 28.4865                         & 37.3011                         \\
            3 & col2 text          & col3 text          & 0.7426                          & 1.0393                          & 1.2507                          & 0.9095                          & 1.2524                          & 1.4958                          \\
            &                    & col3 text          & 12.8008                         & 19.9620                         & 26.0324                         & 16.6347                         & 26.0843                         & 34.0765                         \\
            & col2 text          & col3 text          & 0.7285                          & 1.0257                          & 1.2374                          & 0.8195                          & 1.1407                          & 1.3691\tnote{*}                 \\
            &                    & col3 text          & 13.0360                         & 20.2690                         & 26.3895                         & 15.0812                         & 23.4932                         & 30.6060\tnote{\dagger}          \\
            \bottomrule
        \end{tabular*}
        \begin{tablenotes}%%[\textheight]
            \item[*] First sideways table footnote. Sideways table footnote. Sideways table footnote. Sideways table footnote.
            \item[$\dagger$] Second sideways table footnote. Sideways table footnote. Sideways table footnote. Sideways table footnote.
        \end{tablenotes}
    \end{sidewaystable}

    \begin{sidewaysfigure}
        \centerline{\includegraphics[width=542pt,height=9pc,draft]{empty}}
        \caption{Sideways figure caption. Sideways figure caption. Sideways figure caption. Sideways figure caption. Sideways figure caption. Sideways figure caption.\label{fig3}}
    \end{sidewaysfigure}

    nibh. Pellentesque wisi. Nullam malesuada. Morbi ut tellus ut pede tincidunt porta. Lorem ipsum dolor sit amet,
    consectetuer adipiscing elit. Etiam congue neque id dolor.

    \begin{algorithm}
        \caption{Pseudocode for our algorithm}\label{alg1}
        \begin{algorithmic}
            \For
                each frame
                \For
                    water particles $f_{i}$
                    \State compute fluid flow~\cite{Rothermel1997}
                    \State compute fluid--solid interaction~\cite{Allen2011}
                    \State apply adhesion and surface tension~\cite{Ballen2011}
                \EndFor
                \For
                    solid particles $s_{i}$
                    \For
                        neighboring water particles $f_{j}$
                        \State compute virtual water film \\(see Section~\ref{sec3})
                    \EndFor
                \EndFor
                \For
                    solid particles $s_{i}$
                    \For
                        neighboring water particles $f_{j}$
                        \State compute growth direction vector \\(see Section~\ref{sec2})
                    \EndFor
                \EndFor
                \For
                    solid particles $s_{i}$
                    \For
                        neighboring water particles $f_{j}$
                        \State compute $F_{\theta}$ (see Section~\ref{sec1})
                        \State compute $CE(s_{i},f_{j})$ \\(see Section~\ref{sec3})
                        \If
                            $CE(b_{i}, f_{j})$ $>$ glaze threshold
                            \State $j$th water particle's phase $\Leftarrow$ ICE
                        \EndIf
                        \If
                            $CE(c_{i}, f_{j})$ $>$ icicle threshold
                            \State $j$th water particle's phase $\Leftarrow$ ICE
                        \EndIf
                    \EndFor
                \EndFor
            \EndFor
        \end{algorithmic}
    \end{algorithm}

    Donec et nisl at wisi luctus bibendum. Nam interdum tellus ac libero. Sed sem justo, laoreet vitae, fringilla at,
    adipiscing ut, nibh. Maecenas non sem quis tortor eleifend fermentum. Etiam id tortor ac mauris porta vulputate.
    Integer porta neque vitae massa~\cite{Rothermel1997,Elbaum2002}. Maecenas tempus libero a libero posuere dictum. Vestibulum ante ipsum primis in
    faucibus orci luctus et ultrices posuere cubilia Curae; Aenean quis mauris sed elit commodo placerat. Class aptent
    taciti sociosqu ad litora torquent per conubia nostra, per inceptos hymenaeos. Vivamus rhoncus tincidunt libero.
    Etiam elementum pretium justo. Vivamus est. Morbi a tellus eget pede tristique commodo~\cite{Elbaum2002} Nulla nisl. Vestibulum
    sed nisl eu sapien cursus rutrum.

    Pellentesque wisi. Nullam malesuada. Morbi ut tellus ut pede tincidunt porta. Lorem ipsum dolor sit amet,
    consectetuer adipiscing elit. Etiam congue neque id dolor.

    Donec et nisl at wisi luctus bibendum. Nam interdum tellus ac libero. Sed sem justo, laoreet vitae, fringilla at,
    adipiscing ut, nibh. Maecenas non sem quis tortor eleifend fermentum. Etiam id tortor ac mauris porta vulputate.
    Integer porta neque vitae massa. Maecenas tempus libero a libero posuere dictum. Vestibulum ante ipsum primis in
    faucibus orci luctus et ultrices posuere cubilia Curae; Aenean quis mauris sed elit commodo placerat. Class aptent
    taciti sociosqu ad litora torquent per conubia nostra, per inceptos hymenaeos. Vivamus rhoncus tincidunt libero.
    Etiam elementum pretium justo. Vivamus est. Morbi a tellus eget pede tristique commodo. Nulla nisl. Vestibulum
    sed nisl eu sapien cursus rutrum.

    \begin{equation}
        \label{eq23}
        \|\tilde{X}(k)\|^2
        =\frac{\left\|\sum\limits_{i=1}^{p}\tilde{Y}_i(k)+\sum\limits_{j=1}^{q}\tilde{Z}_j(k) \right\|^2}{(p+q)^2}
        \leq\frac{\sum\limits_{i=1}^{p}\left\|\tilde{Y}_i(k)\right\|^2+\sum\limits_{j=1}^{q}\left\|\tilde{Z}_j(k)\right\|^2 }{p+q}.
    \end{equation}

    Sed feugiat. Cum sociis natoque penatibus et magnis dis parturient montes, nascetur ridiculus mus. Ut pellentesque
    augue sed urna. Vestibulum diam eros, fringilla et, consectetuer eu, nonummy id, sapien. Nullam at lectus. In sagittis
    ultrices mauris. Curabitur malesuada erat sit amet massa. Fusce blandit. Aliquam erat volutpat. Aliquam euismod.
    Aenean vel lectus. Nunc imperdiet justo nec dolor.

    Etiam euismod. Fusce facilisis lacinia dui. Suspendisse potenti. In mi erat, cursus id, nonummy sed, ullamcorper
    eget, sapien. Praesent pretium, magna in eleifend egestas, pede pede pretium lorem, quis consectetuer tortor sapien
    facilisis magna. Mauris quis magna varius nulla scelerisque imperdiet. Aliquam non quam. Aliquam porttitor quam
    a lacus. Praesent vel arcu ut tortor cursus volutpat. In vitae pede quis diam bibendum placerat. Fusce elementum
    convallis neque. Sed dolor orci, scelerisque ac, dapibus nec, ultricies ut, mi. Duis nec dui quis leo sagittis commodo.

    \begin{equation}
        \label{eq24}
        \|\tilde{X}(k)\|^2
        =\frac{\left\|\sum\limits_{i=1}^{p}\tilde{Y}_i(k)+\sum\limits_{j=1}^{q}\tilde{Z}_j(k) \right\|^2}{(p+q)^2}
        \leq\frac{\sum\limits_{i=1}^{p}\left\|\tilde{Y}_i(k)\right\|^2+\sum\limits_{j=1}^{q}\left\|\tilde{Z}_j(k)\right\|^2 }{p+q}.
    \end{equation}

    Aliquam lectus. Vivamus leo. Quisque ornare tellus ullamcorper nulla. Mauris porttitor pharetra
    tortor. Sed fringilla justo sed mauris. Mauris tellus. Sed non leo. Nullam elementum, magna in cursus sodales, augue
    est scelerisque sapien, venenatis congue nulla arcu et pede. Ut suscipit enim vel sapien. Donec congue. Maecenas
    urna mi, suscipit in, placerat ut, vestibulum ut, massa. Fusce ultrices nulla et nisl.

    Etiam ac leo a risus tristique nonummy. Donec dignissim tincidunt nulla. Vestibulum rhoncus molestie odio. Sed
    lobortis, justo et pretium lobortis, mauris turpis condimentum augue, nec ultricies nibh arcu pretium enim. Nunc
    purus neque, placerat id, imperdiet sed, pellentesque nec, nisl. Vestibulum imperdiet neque non sem accumsan laoreet.
    In hac habitasse platea dictumst. Etiam condimentum facilisis libero. Suspendisse in elit quis nisl aliquam dapibus.
    Pellentesque auctor sapien. Sed egestas sapien nec lectus. Pellentesque vel dui vel neque bibendum viverra. Aliquam
    porttitor nisl nec pede. Proin mattis libero vel turpis. Donec rutrum mauris et libero. Proin euismod porta felis.
    Nam lobortis, metus quis elementum commodo, nunc lectus elementum mauris, eget vulputate ligula tellus eu neque.
    Vivamus eu dolor.


    \section{Conclusions}\label{sec5}

    Lorem ipsum dolor sit amet, consectetuer adipiscing elit. Ut purus elit, vestibulum ut, placerat ac, adipiscing vitae,
    felis. Curabitur dictum gravida mauris. Nam arcu libero, nonummy eget, consectetuer id, vulputate a, magna. Donec
    vehicula augue eu neque. Pellentesque habitant morbi tristique senectus et netus et malesuada fames ac turpis egestas.
    Mauris ut leo. Cras viverra metus rhoncus sem. Nulla et lectus vestibulum urna fringilla ultrices. Phasellus eu tellus
    sit amet tortor gravida placerat. Integer sapien est, iaculis in, pretium quis, viverra ac, nunc. Praesent eget sem vel
    leo ultrices bibendum. Aenean faucibus. Morbi dolor nulla, malesuada eu, pulvinar at, mollis ac, nulla. Curabitur
    auctor semper nulla. Donec varius orci eget risus. Duis nibh mi, congue eu, accumsan eleifend, sagittis quis, diam.
    Duis eget orci sit amet orci dignissim rutrum.

    Nam dui ligula, fringilla a, euismod sodales, sollicitudin vel, wisi. Morbi auctor lorem non justo. Nam lacus libero,
    pretium at, lobortis vitae, ultricies et, tellus. Donec aliquet, tortor sed accumsan bibendum, erat ligula aliquet magna,
    vitae ornare odio metus a mi. Morbi ac orci et nisl hendrerit mollis. Suspendisse ut massa. Cras nec ante. Pellentesque
    a nulla. Cum sociis natoque penatibus et magnis dis parturient montes, nascetur ridiculus mus. Aliquam tincidunt
    urna. Nulla ullamcorper vestibulum turpis. Pellentesque cursus luctus mauris.

%\backmatter

    \section*{Acknowledgments}
    This is acknowledgment text~\cite{Elbaum2002}. Provide text here. This is acknowledgment text. Provide text here. This is acknowledgment text. Provide text here. This is acknowledgment text. Provide text here. This is acknowledgment text. Provide text here. This is acknowledgment text. Provide text here. This is acknowledgment text. Provide text here. This is acknowledgment text. Provide text here. This is acknowledgment text. Provide text here.

    \subsection*{Author contributions}

    This is an author contribution text. This is an author contribution text. This is an author contribution text. This is an author contribution text. This is an author contribution text.

    \subsection*{Financial disclosure}

    None reported.

    \subsection*{Conflict of interest}

    The authors declare no potential conflict of interests.


    \section*{Supporting information}

    The following supporting information is available as part of the online article:

    \noindent
    \textbf{Figure S1.}
    {500{\uns}hPa geopotential anomalies for GC2C calculated against the ERA Interim reanalysis. The period is 1989--2008.}

    \noindent
    \textbf{Figure S2.}
    {The SST anomalies for GC2C calculated against the observations (OIsst).}


    \appendix


    \section{Section title of first appendix\label{app1}}

    Use \verb+\begin{verbatim}...\end{verbatim}+ for program codes without math. Use \verb+\begin{alltt}...\end{alltt}+ for program codes with math. Based on the text provided inside the optional argument of \verb+\begin{code}[Psecode|Listing|Box|Code|+\hfill\break \verb+Specification|Procedure|Sourcecode|Program]...+ \verb+\end{code}+ tag corresponding boxed like floats are generated. Also note that \verb+\begin{code}[Code|Listing]...+ \verb+\end{code}+ tag with either Code or Listing text as optional argument text are set with computer modern typewriter font. All other code environments are set with normal text font. Refer below example:

    \begin{lstlisting}[caption={Descriptive Caption Text},label=DescriptiveLabel]
for i:=maxint to 0 do
begin
{ do nothing }
end;
Write('Case insensitive ');
WritE('Pascal keywords.');
    \end{lstlisting}

    \subsection{Subsection title of first appendix\label{app1.1a}}

    Nam dui ligula, fringilla a, euismod sodales, sollicitudin vel, wisi. Morbi auctor lorem non justo. Nam lacus libero,
    pretium at, lobortis vitae, ultricies et, tellus. Donec aliquet, tortor sed accumsan bibendum, erat ligula aliquet magna,
    vitae ornare odio metus a mi. Morbi ac orci et nisl hendrerit mollis. Suspendisse ut massa. Cras nec ante. Pellentesque
    a nulla. Cum sociis natoque penatibus et magnis dis parturient montes, nascetur ridiculus mus. Aliquam tincidunt
    urna. Nulla ullamcorper vestibulum turpis. Pellentesque cursus luctus mauris.

    Nulla malesuada porttitor diam. Donec felis erat, congue non, volutpat at, tincidunt tristique, libero. Vivamus
    viverra fermentum felis. Donec nonummy pellentesque ante. Phasellus adipiscing semper elit. Proin fermentum massa
    ac quam. Sed diam turpis, molestie vitae, placerat a, molestie nec, leo. Maecenas lacinia. Nam ipsum ligula, eleifend
    at, accumsan nec, suscipit a, ipsum. Morbi blandit ligula feugiat magna. Nunc eleifend consequat lorem. Sed lacinia
    nulla vitae enim. Pellentesque tincidunt purus vel magna. Integer non enim. Praesent euismod nunc eu purus. Donec
    bibendum quam in tellus. Nullam cursus pulvinar lectus. Donec et mi. Nam vulputate metus eu enim. Vestibulum
    pellentesque felis eu massa.

    \subsubsection{Subsection title of first appendix\label{app1.1.1a}}

    \noindent\textbf{Unnumbered figure}


    \begin{center}
        \includegraphics[width=7pc,height=8pc,draft]{empty}
    \end{center}


    Fusce mauris. Vestibulum luctus nibh at lectus. Sed bibendum, nulla a faucibus semper, leo velit ultricies tellus, ac
    venenatis arcu wisi vel nisl. Vestibulum diam. Aliquam pellentesque, augue quis sagittis posuere, turpis lacus congue
    quam, in hendrerit risus eros eget felis. Maecenas eget erat in sapien mattis porttitor. Vestibulum porttitor. Nulla
    facilisi. Sed a turpis eu lacus commodo facilisis. Morbi fringilla, wisi in dignissim interdum, justo lectus sagittis dui, et
    vehicula libero dui cursus dui. Mauris tempor ligula sed lacus. Duis cursus enim ut augue. Cras ac magna. Cras nulla.

    Nulla egestas. Curabitur a leo. Quisque egestas wisi eget nunc. Nam feugiat lacus vel est. Curabitur consectetuer.
    Suspendisse vel felis. Ut lorem lorem, interdum eu, tincidunt sit amet, laoreet vitae, arcu. Aenean faucibus pede eu
    ante. Praesent enim elit, rutrum at, molestie non, nonummy vel, nisl. Ut lectus eros, malesuada sit amet, fermentum
    eu, sodales cursus, magna. Donec eu purus. Quisque vehicula, urna sed ultricies auctor, pede lorem egestas dui, et
    convallis elit erat sed nulla. Donec luctus. Curabitur et nunc. Aliquam dolor odio, commodo pretium, ultricies non,
    pharetra in, velit. Integer arcu est, nonummy in, fermentum faucibus, egestas vel, odio.


    \section{Section title of second appendix\label{app2}}%

    Fusce mauris. Vestibulum luctus nibh at lectus. Sed bibendum, nulla a faucibus semper, leo velit ultricies tellus, ac
    venenatis arcu wisi vel nisl. Vestibulum diam. Aliquam pellentesque, augue quis sagittis posuere, turpis lacus congue
    quam, in hendrerit risus eros eget felis. Maecenas eget erat in sapien mattis porttitor. Vestibulum porttitor. Nulla
    facilisi. Sed a turpis eu lacus commodo facilisis. Morbi fringilla, wisi in dignissim interdum, justo lectus sagittis dui, et
    vehicula libero dui cursus dui. Mauris tempor ligula sed lacus. Duis cursus enim ut augue. Cras ac magna. Cras nulla.

    Nulla egestas. Curabitur a leo. Quisque egestas wisi eget nunc. Nam feugiat lacus vel est. Curabitur consectetuer.
    Suspendisse vel felis. Ut lorem lorem, interdum eu, tincidunt sit amet, laoreet vitae, arcu. Aenean faucibus pede eu
    ante. Praesent enim elit, rutrum at, molestie non, nonummy vel, nisl. Ut lectus eros, malesuada sit amet, fermentum
    eu, sodales cursus, magna. Donec eu purus. Quisque vehicula, urna sed ultricies auctor, pede lorem egestas dui, et
    convallis elit erat sed nulla. Donec luctus. Curabitur et nunc. Aliquam dolor odio, commodo pretium, ultricies non,
    pharetra in, velit. Integer arcu est, nonummy in, fermentum faucibus, egestas vel, odio.

%== Figure 4 ==
%% Example for figure inside appendix
    \begin{figure}[t]
        \centerline{\includegraphics[height=10pc,width=78mm,draft]{empty}}
        \caption{This is an example for appendix figure.\label{fig5}}
    \end{figure}

    \subsection{Subsection title of second appendix\label{app2.1a}}

    Sed commodo posuere pede. Mauris ut est. Ut quis purus. Sed ac odio. Sed vehicula hendrerit sem. Duis non odio.
    Morbi ut dui. Sed accumsan risus eget odio. In hac habitasse platea dictumst. Pellentesque non elit. Fusce sed justo
    eu urna porta tincidunt. Mauris felis odio, sollicitudin sed, volutpat a, ornare ac, erat. Morbi quis dolor. Donec
    pellentesque, erat ac sagittis semper, nunc dui lobortis purus, quis congue purus metus ultricies tellus. Proin et quam.
    Class aptent taciti sociosqu ad litora torquent per conubia nostra, per inceptos hymenaeos. Praesent sapien turpis,
    fermentum vel, eleifend faucibus, vehicula eu, lacus.

    Pellentesque habitant morbi tristique senectus et netus et malesuada fames ac turpis egestas. Donec odio elit,
    dictum in, hendrerit sit amet, egestas sed, leo. Praesent feugiat sapien aliquet odio. Integer vitae justo. Aliquam
    vestibulum fringilla lorem. Sed neque lectus, consectetuer at, consectetuer sed, eleifend ac, lectus. Nulla facilisi.
    Pellentesque eget lectus. Proin eu metus. Sed porttitor. In hac habitasse platea dictumst. Suspendisse eu lectus. Ut
    mi mi, lacinia sit amet, placerat et, mollis vitae, dui. Sed ante tellus, tristique ut, iaculis eu, malesuada ac, dui.
    Mauris nibh leo, facilisis non, adipiscing quis, ultrices a, dui.

    \subsubsection{Subsection title of second appendix\label{app2.1.1a}}

    Lorem ipsum dolor sit amet, consectetuer adipiscing elit. Ut purus elit, vestibulum ut, placerat ac, adipiscing vitae,
    felis. Curabitur dictum gravida mauris. Nam arcu libero, nonummy eget, consectetuer id, vulputate a, magna. Donec
    vehicula augue eu neque. Pellentesque habitant morbi tristique senectus et netus et malesuada fames ac turpis egestas.
    Mauris ut leo. Cras viverra metus rhoncus sem. Nulla et lectus vestibulum urna fringilla ultrices. Phasellus eu tellus
    sit amet tortor gravida placerat. Integer sapien est, iaculis in, pretium quis, viverra ac, nunc. Praesent eget sem vel
    leo ultrices bibendum. Aenean faucibus. Morbi dolor nulla, malesuada eu, pulvinar at, mollis ac, nulla. Curabitur
    auctor semper nulla. Donec varius orci eget risus. Duis nibh mi, congue eu, accumsan eleifend, sagittis quis, diam.
    Duis eget orci sit amet orci dignissim rutrum.

    Nam dui ligula, fringilla a, euismod sodales, sollicitudin vel, wisi. Morbi auctor lorem non justo. Nam lacus libero,
    pretium at, lobortis vitae, ultricies et, tellus. Donec aliquet, tortor sed accumsan bibendum, erat ligula aliquet magna,
    vitae ornare odio metus a mi. Morbi ac orci et nisl hendrerit mollis. Suspendisse ut massa. Cras nec ante. Pellentesque
    a nulla. Cum sociis natoque penatibus et magnis dis parturient montes, nascetur ridiculus mus. Aliquam tincidunt
    urna. Nulla ullamcorper vestibulum turpis. Pellentesque cursus luctus mauris.

    \begin{center}
        \begin{table*}[b]%
            \centering
            \caption{This is an example of Appendix table showing food requirements of army, navy and airforce.\label{tab4}}%
            \begin{tabular*}{300pt}{@{\extracolsep\fill}lcc@{\extracolsep\fill}}%
                \toprule
                \textbf{col1 head} & \textbf{col2 head} & \textbf{col3 head} \\
                \midrule
                col1 text          & col2 text          & col3 text          \\
                col1 text          & col2 text          & col3 text          \\
                col1 text          & col2 text          & col3 text          \\
                \bottomrule
            \end{tabular*}
        \end{table*}
    \end{center}


    Example for an equation inside appendix
    \begin{equation}
        \mathcal{L}\quad \mathbf{\mathcal{L}} = i \bar{\psi} \gamma^\mu D_\mu \psi - \frac{1}{4} F_{\mu\nu}^a F^{a\mu\nu} - m \bar{\psi} \psi\label{eq25}
    \end{equation}


    \section{Example of another appendix section\label{app3}}%

    This is sample for paragraph text this is sample for paragraph text this is sample for paragraph text this is sample for paragraph text this is sample for paragraph text this is sample for paragraph text this is sample for paragraph text this is sample for paragraph text this is sample for paragraph text this is sample for paragraph text this is sample for paragraph text this is sample for paragraph text this is sample for paragraph text this is sample for paragraph text this is sample for paragraph text this is sample for paragraph text this is sample for paragraph text this is sample for paragraph text this is sample for paragraph text this is sample for paragraph text this is sample for paragraph text this is sample for paragraph text this is sample for paragraph text this is sample for paragraph text this is sample for paragraph text this is sample for paragraph text this is sample for paragraph text this is sample for paragraph text this is sample for paragraph text this is sample for paragraph text this is sample for paragraph text this is sample for paragraph text this is sample for paragraph text


    Nam dui ligula, fringilla a, euismod sodales, sollicitudin vel, wisi. Morbi auctor lorem non justo. Nam lacus libero,
    pretium at, lobortis vitae, ultricies et, tellus. Donec aliquet, tortor sed accumsan bibendum, erat ligula aliquet magna,
    vitae ornare odio metus a mi. Morbi ac orci et nisl hendrerit mollis. Suspendisse ut massa. Cras nec ante. Pellentesque
    a nulla. Cum sociis natoque penatibus et magnis dis parturient montes, nascetur ridiculus mus. Aliquam tincidunt
    urna. Nulla ullamcorper vestibulum turpis. Pellentesque cursus luctus mauris.

    Nulla malesuada porttitor diam. Donec felis erat, congue non, volutpat at, tincidunt tristique, libero. Vivamus
    viverra fermentum felis. Donec nonummy pellentesque ante. Phasellus adipiscing semper elit. Proin fermentum massa
    ac quam. Sed diam turpis, molestie vitae, placerat a, molestie nec, leo. Maecenas lacinia. Nam ipsum ligula, eleifend
    at, accumsan nec, suscipit a, ipsum. Morbi blandit ligula feugiat magna. Nunc eleifend consequat lorem. Sed lacinia
    nulla vitae enim. Pellentesque tincidunt purus vel magna. Integer non enim. Praesent euismod nunc eu purus. Donec
    bibendum quam in tellus. Nullam cursus pulvinar lectus. Donec et mi. Nam vulputate metus eu enim. Vestibulum
    pellentesque felis eu massa.


    \begin{equation}
        \mathcal{L} = i \bar{\psi} \gamma^\mu D_\mu \psi
        - \frac{1}{4} F_{\mu\nu}^a F^{a\mu\nu} - m \bar{\psi} \psi
        \label{eq26}
    \end{equation}

    Nulla malesuada porttitor diam. Donec felis erat, congue non, volutpat at, tincidunt tristique, libero. Vivamus
    viverra fermentum felis. Donec nonummy pellentesque ante. Phasellus adipiscing semper elit. Proin fermentum massa
    ac quam. Sed diam turpis, molestie vitae, placerat a, molestie nec, leo. Maecenas lacinia. Nam ipsum ligula, eleifend
    at, accumsan nec, suscipit a, ipsum. Morbi blandit ligula feugiat magna. Nunc eleifend consequat lorem. Sed lacinia
    nulla vitae enim. Pellentesque tincidunt purus vel magna. Integer non enim. Praesent euismod nunc eu purus. Donec
    bibendum quam in tellus. Nullam cursus pulvinar lectus. Donec et mi. Nam vulputate metus eu enim. Vestibulum
    pellentesque felis eu massa.

    Quisque ullamcorper placerat ipsum. Cras nibh. Morbi vel justo vitae lacus tincidunt ultrices. Lorem ipsum dolor sit
    amet, consectetuer adipiscing elit. In hac habitasse platea dictumst. Integer tempus convallis augue. Etiam facilisis.
    Nunc elementum fermentum wisi. Aenean placerat. Ut imperdiet, enim sed gravida sollicitudin, felis odio placerat
    quam, ac pulvinar elit purus eget enim. Nunc vitae tortor. Proin tempus nibh sit amet nisl. Vivamus quis tortor
    vitae risus porta vehicula.


    \begin{center}
        \begin{tabular*}{250pt}{@{\extracolsep\fill}lcc@{\extracolsep\fill}}%
            \toprule
            \textbf{col1 head} & \textbf{col2 head} & \textbf{col3 head} \\
            \midrule
            col1 text          & col2 text          & col3 text          \\
            col1 text          & col2 text          & col3 text          \\
            col1 text          & col2 text          & col3 text          \\
            \bottomrule
        \end{tabular*}
    \end{center}


    Quisque ullamcorper placerat ipsum. Cras nibh. Morbi vel justo vitae lacus tincidunt ultrices. Lorem ipsum dolor sit
    amet, consectetuer adipiscing elit. In hac habitasse platea dictumst. Integer tempus convallis augue. Etiam facilisis.
    Nunc elementum fermentum wisi. Aenean placerat. Ut imperdiet, enim sed gravida sollicitudin, felis odio placerat
    quam, ac pulvinar elit purus eget enim. Nunc vitae tortor. Proin tempus nibh sit amet nisl. Vivamus quis tortor
    vitae risus porta vehicula.

    Fusce mauris. Vestibulum luctus nibh at lectus. Sed bibendum, nulla a faucibus semper, leo velit ultricies tellus, ac
    venenatis arcu wisi vel nisl. Vestibulum diam. Aliquam pellentesque, augue quis sagittis posuere, turpis lacus congue
    quam, in hendrerit risus eros eget felis. Maecenas eget erat in sapien mattis porttitor. Vestibulum porttitor. Nulla
    facilisi. Sed a turpis eu lacus commodo facilisis. Morbi fringilla, wisi in dignissim interdum, justo lectus sagittis dui, evehicula libero dui cursus dui. Mauris tempor ligula sed lacus. Duis cursus enim ut augue. Cras ac magna. Cras nulla.
    Nulla egestas. Curabitur a leo. Quisque egestas wisi eget nunc. Nam feugiat lacus vel est. Curabitur consectetuer.

    Pellentesque habitant morbi tristique senectus et netus et malesuada fames ac turpis egestas. Donec odio elit,
    dictum in, hendrerit sit amet, egestas sed, leo. Praesent feugiat sapien aliquet odio. Integer vitae justo. Aliquam
    vestibulum fringilla lorem. Sed neque lectus, consectetuer at, consectetuer sed, eleifend ac, lectus. Nulla facilisi.
    Pellentesque eget lectus. Proin eu metus. Sed porttitor. In hac habitasse platea dictumst. Suspendisse eu lectus. Ut
    mi mi, lacinia sit amet, placerat et, mollis vitae, dui. Sed ante tellus, tristique ut, iaculis eu, malesuada ac, dui.
    Mauris nibh leo, facilisis non, adipiscing quis, ultrices a, dui.

    \nocite{*}% Show all bib entries - both cited and uncited; comment this line to view only cited bib entries;
    \bibliography{wileyNJD-AMS}%


    \section*{Author Biography}

    \begin{biography}{\includegraphics[width=60pt,height=70pt,draft]{empty}}{\textbf{Author Name.} This is sample author biography text this is sample author biography text this is sample author biography text this is sample author biography text this is sample author biography text this is sample author biography text this is sample author biography text this is sample author biography text this is sample author biography text this is sample author biography text this is sample author biography text this is sample author biography text this is sample author biography text this is sample author biography text this is sample author biography text this is sample author biography text this is sample author biography text this is sample author biography text this is sample author biography text this is sample author biography text this is sample author biography text.}
    \end{biography}

\end{document}
